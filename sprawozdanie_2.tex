\documentclass{article}
\usepackage{amsmath}
\usepackage{graphicx}
\usepackage{hyperref}
\usepackage[utf8]{inputenc}

\usepackage{listings}
\usepackage{color}

\NewDocumentCommand{\codeword}{v}{
\texttt{\textcolor{black}{#1}}
}
\NewDocumentCommand{\code}{v}{
\texttt{\textcolor{black}{#1}}
}

\definecolor{dkgreen}{rgb}{0,0.6,0}
\definecolor{gray}{rgb}{0.5,0.5,0.5}
\definecolor{mauve}{rgb}{0.58,0,0.82}

\lstset{frame=tb,
  aboveskip=3mm,
  belowskip=3mm,
  showstringspaces=false,
  columns=flexible,
  basicstyle={\small\ttfamily},
  numbers=none,
  numberstyle=\tiny\color{gray},
  keywordstyle=\color{blue},
  commentstyle=\color{dkgreen},
  stringstyle=\color{mauve},
  breaklines=true,
  breakatwhitespace=true,
  tabsize=3
}


\title{Sprawozdanie nr. 2}
\author{Mateusz Kojro}

\date{2020–01–13}
\begin{document}
\maketitle

\section{Opis cwiczenia}

Celem tego cwiczenia jest poznanie sposobow uzywania i wykorzystanie licznika jako
ukladu mierzacego czasu (Timer0) ale rowniez przyklad generowania sygnalu PWM (Timer1)
Timery w mikrokontrolerze dzialaja w zasadzie licznikow. Wykorzystywany przez nas mikrokontroler
posiada 2 rozne timery 8 i 16 bitowy sa one wiec w stanie odliczac odpowiednio do
256 i 65 536 w rownych odstepach czasu. Posiadajac taki timer jestesmy w stanie
wygenerowac sygnal PWM (Pulse width modulation) ktory dziala na zasadzie 
generowania przebiegu prostokatnego o zmiennych dlugosciach wysylania 0 i 1
sygnal taki umozliwia np. stopniowe przyciemnianie diody LED

\section{Wykorzystane polecenia jezyka BASCOM}
\begin{itemize}
  \item \code{Config Timer0} - Konfiguracja pracy Timer0
  \item \code{Start | Stop} - Umozliwia zatrzymanie i wystartowanie licznika
  \item \code{Counter0 = value} - Ustawia poczatkowa wartosc licznika
  \item \code{Load Timer0 number} - Wartosc po ktore licznik sie przepelni
  \item \code{On Interupt} - Konfiguruje dzialania wykonywane podczas ogloszenia przerwania
  \item \code{Enable | Disable Interupt} - Ustawienie zglaszania przerwan podczas pracy mikrokontrolera
\end{itemize}

\section{Wykorzystywane przyrzady}
\begin{itemize}
  \item Oscyloskop cyfrowy
\end{itemize}

\section{Przeprowadzone doswiadczenia}

Gdy dokonalismy pomiaru okresu przerwan wbudowanego timera teoretycznie ustawionego 
na okers 1 sekundy otrzymalismy okres zmierzony za pomoca oscyloskopu rowny 
968ms. Jak latwo obliczyc w skali roku skonstruowany w ten sposob zegar mylil by sie o ponad 
11 dni. Aby otrzymac wyniki blizsze prawdy nalezy wykorzystac zewnetrzny rezonator kwarcowy.

\begin{lstlisting}[language=VBScript , caption=Program odmierzajacy 1s]
  $regfile = "m8def.dat"
  $crystal = 8000000	

  Config Pinb.0 = Output
  Config Timer0 = TIMER, Prescale = 256
  256
  
  On Timer0 Odmierz_1s
  
  Dim Licz_8ms As Byte
  
  Enable Interrupts
  Enable Timer0	
  Load Timer0 = 250
  
  Do
  Loop
  End
  
  Odmierz_1s:
        Load Timer0 =250
        Incr Licz_8ms
        If Licz_8ms = 125 Then
           Licz_8ms = 0
           Toggle Portb.0
        End If
  Return
\end{lstlisting}

Aby wygenerowac sygnal PWM o czestotliwosci 7.82 kHz i wypelnionego w 60\%

Czestotliwosc:
\begin{equation}
  \begin{split}
    F_c = \frac{\frac{8 * 10^6 Hz}{1}}{1022} = 0.00782 * 10^6 Hz = 7.82 KHz \\
  \end{split}
\end{equation}

Wypelnienie:
\begin{equation}
    k_w = 6 * x = 306.6
\end{equation}

\begin{equation}
  \begin{split}
    511 &\leftrightarrow 100\% \\
    x &\leftrightarrow 10\% 
  \end{split}
\end{equation}

mamy wiec:

\begin{equation}
  x = 511 * \frac{10}{100} = 51.1
\end{equation}


\begin{lstlisting}[language=VBScript , caption=Kod programu generujacego sygnal PWM]
  $regfile = "m8def.dat"
  $crystal = 8000000	
  Config Pinb.1 = output
  Config Timer1 = PWM,            
  PWM = 8, 
  Compare A PWM = Clear Up, 
  Compare B PWM = Disconnect, 
  Prescale = 256
  PWM1a = 127.5
  End
\end{lstlisting}

\end{document}
