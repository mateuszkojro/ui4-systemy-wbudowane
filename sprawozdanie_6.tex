\documentclass{article}
\usepackage{amsmath}
\usepackage{graphicx}
\usepackage{hyperref}
\usepackage[utf8]{inputenc}

\usepackage{listings}
\usepackage{color}

\NewDocumentCommand{\codeword}{v}{
\texttt{\textcolor{black}{#1}}
}

\definecolor{dkgreen}{rgb}{0,0.6,0}
\definecolor{gray}{rgb}{0.5,0.5,0.5}
\definecolor{mauve}{rgb}{0.58,0,0.82}

\lstset{frame=tb,
  aboveskip=3mm,
  belowskip=3mm,
  showstringspaces=false,
  columns=flexible,
  basicstyle={\small\ttfamily},
  numbers=none,
  numberstyle=\tiny\color{gray},
  keywordstyle=\color{blue},
  commentstyle=\color{dkgreen},
  stringstyle=\color{mauve},
  breaklines=true,
  breakatwhitespace=true,
  tabsize=3
}


\title{Sprawozdanie nr. 6}
\author{Mateusz Kojro}


\date{2020–01–13}
\begin{document}
\maketitle

\section{Opis cwiczenia}

Cwiczenie polegalo na pomiarze temperatury oraz sterowaniu ukladami DS1820 za pomoca magistrali 1-wire
Czujniki temperatury realizowane sa korzystajac z roznych zjawisk fizycznych (miedzy innymi odksztalcanie bimetali czy zmiany oporu materialu)
Magistrala 1-wire (w naszym przypadku wykonana przez firme DALLAS) sluzy do przekazywania informacji pomiedzy
ukladami "slave" a mikrokontrolerem. Jak sugeruje jej nazwa magistrala 1-wire sklada sie wylacznie z jednego przewodu
nie liczac masy.

Aby oddczytac temperature za pomoca magistrali 1-wire i czujnika temperatury nalezy wykonac ponizsze czynnosci

\begin{enumerate}
  \item Zerowanie magistrali
  \item Wypisanie id czujnika
  \item Wyslac komende pomiaru temperatury
  \item Zatrzymac wykonywanie programu na maksymalny czas konwersji analogowych wartosci termometru na wartosci cyfrowe
  \item Wyzerowac magistrale
  \item Wyslac id czujnika
  \item Wyslac komede odczytu temperatury
  \item Odcztac temperature
  \item Wyzerowac magistrale
\end{enumerate}

Przy czym nalezy pamietac ze temperatura zapisana jest w postaci 8bitowej liczby z znakiem (1 bit to bit znaku)
i ze id czujnika musi zostac odcztane podczas pisania programu i zapisane do pamieci EPROM

\section{Instrukcje jezyka BASCOM uzywane do komunikacji za pomoca magistrali 1-wire}

\begin{itemize}
  \item \codeword{1wreset} - resetowanie magistrali
  \item \codeword{1write [value]} - wyslij \codeword{value} na magistrale
  \item \codeword{1wread} - odczytaj wartosc z magistralii
  \item \codeword{Writeeeprom} - Zapisz wartosc do pamieci EEPROM
\end{itemize}

\section{Elementy elektroniczne uzyte do realizacji ukladu}
\begin{itemize}
  \item Uklad DS1820
  \item Wyswietlacz LCD
\end{itemize}

\section{Kod odczytujacy temperature za pomoca ukladu DS1820}

\begin{lstlisting}[language=VBScript , caption=Kod odczytujacy temperature za pomoca ukladu DS1820]
  $regfile = "m8def.dat"
  $crystal = 8000000
  Config Lcd = 16 * 2
  Config 1wire = PORTB.0

  Declare Sub read_temp
  Dim temp(2) As Byte

  DefLcdChar 0, 7, 5, 7, 32, 32, 32, 32, 32

  Do
    Call read_temp
    Cls
    If temp(2) = 0 Then
      Lcd "Temp: " ; temp(1) ; Chr(0) ; "C"
    Else
      Lcd "Temp: -" ; temp(1) ; Chr(0) ; "C"
    End If
  Loop
  End

  Sub read_tempe
    1wreset
    1wwrite &hcc
    1wwrite &h44
    Waitms 750
    1wreset
    1wwrite &hcc
    1wwrite &hbe
    temp(2) = 1wread(2)
    1wreset

    If Err = 1 Then
      Cls
      Lcd "Blad"
      Do
      Loop
    End If
    If temp(2) > 0 Then
      temp(1) = 256 - tem(1)
    End If
    temp(1) = temp(1)/2
  End Sub
\end{lstlisting}

\end{document}