
\documentclass{article}
\usepackage{amsmath}
\usepackage{graphicx}
\usepackage{hyperref}
\usepackage[utf8]{inputenc}

\usepackage{listings}
\usepackage{color}

\NewDocumentCommand{\codeword}{v}{
\texttt{\textcolor{black}{#1}}
}
\NewDocumentCommand{\code}{v}{
\texttt{\textcolor{black}{#1}}
}

\definecolor{dkgreen}{rgb}{0,0.6,0}
\definecolor{gray}{rgb}{0.5,0.5,0.5}
\definecolor{mauve}{rgb}{0.58,0,0.82}

\lstset{frame=tb,
  aboveskip=3mm,
  belowskip=3mm,
  showstringspaces=false,
  columns=flexible,
  basicstyle={\small\ttfamily},
  numbers=none,
  numberstyle=\tiny\color{gray},
  keywordstyle=\color{blue},
  commentstyle=\color{dkgreen},
  stringstyle=\color{mauve},
  breaklines=true,
  breakatwhitespace=true,
  tabsize=3
}


\title{Sprawozdanie nr. 3}
\author{Mateusz Kojro}



\date{}
\begin{document}
\maketitle

  \section{Opis cwiczenia}
  Cwiczenie ma na celu zaznajomienie z metodami wyswietlania danych za pomoca 7 segmentowych wyswietlaczy LED
  jak i alfanumerycznymi wyswietlaczami LCD.
  Wyswietlacze LED moga byc sterowane zarowno pojedynczo jak i w sposob multipleksowany.

  \section{Polecenia jezyka BASCOM potrzebne podczas sterowania wyswietlaczami}
    \begin{itemize}
      \item \codeword{Enable value} - odblokowuje \code{value} gdzie \code{value} to Interupts lub Timer
        odblokowujac odpowiednio system perzerwan i wbudowany timer
      \item \code{Config Lcd = 16 * 2} - Ustawia liczbe wierszy i koumn w obslugiwanym wyswietlaczu;
      \item \code{Cls} - czysci zawartosc ekranu LCD
      \item \code{lcd text} - wyswietla zawartosc text na ekranie
      \item \code{Lowerline} - przejdz do wyswietlania kolejnej lini
      \item \code{Shiftlcd Right | Left} - Przesun tekst na wyswietlaczu w prawo lub w lewo
      \item \code{Locate x, y} - ustaw pozycje kursora na x, y
      \item \code{Shiftcursor Right | Left} - Przesun kursor w prawo lub w lewo
      \item \code{Display Off | On} - Uruchom lub wylacz wyswietlacz
    \end{itemize}
  \section{Elementy elektroniczne wykorzystywane podczas cwiczenia}

  \begin{itemize}
    \item 4 wyswietlacze 7 segmentowe LED
    \item Wyswietlacz LCD
    \item Uklad mocy ULN2803A
  \end{itemize}

  \section{Przykladowe programy}

\begin{lstlisting}[language=VBScript , caption=Obsluga klawiatury matrycowej]
  $regfile = "m8def.dat"
  $crystal = 8000000
  
  Config portd =Output
  Config pinb.0=Output
  Config pinb.1=Output
  Config pinb.2=Output
  Config pinb.3=Output
  Config Timer0 = Timer , Prescale = 256
  
  Declare Sub Pobr_znaku(cyfra As Byte)
  On Timer0 Mult_wysw
  
  Dim A As Byte , B As Byte , C As Byte , D As Byte
  Dim Ia As Byte, Ib As Byte, Ic as Byte, Id as Byte
  
  Dim Nr_wysw As Byte
  Dim I As Byte
  W1 Alias PORTB.0
  W2 Alias PORTB.1
  W3 Alias PORTB.2
  W4 Alias PORTB.3
  
  Enable Interrupts
  Enable Timer0
  Load Timer0 , 125
  
  Do
  A=9 : B=9 : C=9 : D=9
  
  FOR Ia=9 to 0 Step -1
  A=Ia
  
     FOR Ib=9 to 0 Step -1
    `B=Ib
  
        FOR Ic=9 to 0 Step -1
        C=Ic
  
           FOR Id=9 to 0 Step -1
           D=Id
           wait 1
           Next Id
           Next Ic
           Next Ib
           Next Ia
  
  A=&B01001001 : B=&B10000110 : C=&B11000111 : D=&B11000000
  wait 5
  A=9 : B=9 : C=9 : D=9
  
  Loop
  
  End
  
  Sub Pobr_znaku(cyfra As Byte)
     If Cyfra <10 Then
        Portd = Lookup(cyfra , Kody7seg)
     Else
        portd = 0
     End If
  End Sub
  
  Mult_wysw:
     Load Timer0 , 125
     Set W1
     Set W2
     Set W3
     Set W4
     Select Case Nr_wysw
  
       Case 0:
        Call Pobr_znaku(a)
         Reset W1
       Case 1:
        Call Pobr_znaku(b)
        Reset W2
       Case 2:
        Call Pobr_znaku(c)
        Reset W3
       Case 3:
        Call Pobr_znaku(d)
        Reset W4
     End Select
     Incr Nr_wysw
  
     If Nr_wysw = 4 Then
        Nr_wysw = 0
     End If
  Return
  
  Kody7seg:
  Data &B11000000 , &B11111001 , &B10100100 , &B10110000 , &B10011001
  Data &B10010010 , &B10000010 , &B11111000 , &B10000000 , &B10010000
  Data &B10001001 , &B11000000 , &B11000111 , &B10000110
\end{lstlisting}

\section{Podsumowanie}

 Podczas konstruowania ukladow opartych na  mikrokontrolerach 2 glowne sposoby wyswietlania informacji
 to wyswietlacze 7 segmentowe oparte (umozliwiaja wyswietlenie cyfr i niektorych lter) lub wyswietlacze LCD ktore umozliwiaja
 natomisat wyswietlenie szerokiego zakresu symboli i liter lacznie z mozliwoscia tworzenia personalizowanych


\end{document}
