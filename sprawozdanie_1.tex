\documentclass{article}
\usepackage{amsmath}
\usepackage{graphicx}
\usepackage{hyperref}
\usepackage[utf8]{inputenc}

\usepackage{listings}
\usepackage{color}

\NewDocumentCommand{\codeword}{v}{
\texttt{\textcolor{black}{#1}}
}
\NewDocumentCommand{\code}{v}{
\texttt{\textcolor{black}{#1}}
}

\definecolor{dkgreen}{rgb}{0,0.6,0}
\definecolor{gray}{rgb}{0.5,0.5,0.5}
\definecolor{mauve}{rgb}{0.58,0,0.82}

\lstset{frame=tb,
  aboveskip=3mm,
  belowskip=3mm,
  showstringspaces=false,
  columns=flexible,
  basicstyle={\small\ttfamily},
  numbers=none,
  numberstyle=\tiny\color{gray},
  keywordstyle=\color{blue},
  commentstyle=\color{dkgreen},
  stringstyle=\color{mauve},
  breaklines=true,
  breakatwhitespace=true,
  tabsize=3
}


\title{Sprawozdanie nr. 1}
\author{Mateusz Kojro}

\date{2020–01–13}
\begin{document}
\maketitle

\section{Opis cwiczenia}
  Celem cwiczenia jest porownanie czasu wykonywania instrukcji w jezyku programowania BASCOM
  policzymy rowniez ilosc cykli pracy mikrokontrolera przypadajacych na instrukcje.
  Porownamy nastepnie te wyniki z wynikami dla programow napisancyh w ASSEMBLERZE 



\section{Wykorzystane polecenia jezyka BASCOM}

\begin{itemize}
\item \code{$regfile = ”m8def.dat”} - informacja o stosowanych dyrektywach mikrokontrolera
\item \code{$crystal = 8000000} - informacja dla kompilatora o czestotliwosci zegara kontrolera 
\item \code{CONFIG PINB.0 = Output} - Ustaw PB0 jako linie wejsciowa
\item \code{Do} - Poczatek petli programu
\item \code{SET} - Ustaw wartosc na porcie 1
\item \code{RESET} - Ustaw wartosc na porcie na 0
\item \code{TOGGLE} - zamien 1 na 0 i 0 na 1
\item \code{ROTATE PORTB.LEFT} - przesun wartosci w \code{PORTB} w lewo
\item \code{WAITMS 200} - opoznij dzialanie programu o 200ms
\item \code{LOOP} - Koniec glownej petli programu
\item \code{END} - Koniec programu 
\end{itemize}

\section{Wykorzystywane przyrzady}
\begin{itemize}
  \item Oscyloskop cyforwy
\end{itemize}

\begin{lstlisting}[language=VBScript , caption=Kod programu generujacego sygnal PWM]

\end{lstlisting}

\end{document}
