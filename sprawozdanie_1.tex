\documentclass{article}
\usepackage{amsmath}
\usepackage{graphicx}
\usepackage{hyperref}
\usepackage[utf8]{inputenc}

\usepackage{listings}
\usepackage{color}

\NewDocumentCommand{\codeword}{v}{
\texttt{\textcolor{black}{#1}}
}
\NewDocumentCommand{\code}{v}{
\texttt{\textcolor{black}{#1}}
}

\definecolor{dkgreen}{rgb}{0,0.6,0}
\definecolor{gray}{rgb}{0.5,0.5,0.5}
\definecolor{mauve}{rgb}{0.58,0,0.82}

\lstset{frame=tb,
  aboveskip=3mm,
  belowskip=3mm,
  showstringspaces=false,
  columns=flexible,
  basicstyle={\small\ttfamily},
  numbers=none,
  numberstyle=\tiny\color{gray},
  keywordstyle=\color{blue},
  commentstyle=\color{dkgreen},
  stringstyle=\color{mauve},
  breaklines=true,
  breakatwhitespace=true,
  tabsize=3
}


\title{Sprawozdanie nr. 1}
\author{Mateusz Kojro}

\date{}
\begin{document}
\maketitle

\section{Opis cwiczenia}
  Celem cwiczenia jest porownanie czasu wykonywania instrukcji w jezyku programowania BASCOM
  policzymy rowniez ilosc cykli pracy mikrokontrolera przypadajacych na instrukcje.
  Porownamy nastepnie te wyniki z wynikami dla programow napisancyh w ASSEMBLERZE 



\section{Wykorzystane polecenia jezyka BASCOM}

\begin{itemize}
\item \code{$regfile = ”m8def.dat”} - informacja o stosowanych dyrektywach mikrokontrolera
\item \code{$crystal = 8000000} - informacja dla kompilatora o czestotliwosci zegara kontrolera 
\item \code{CONFIG PINB.0 = Output} - Ustaw PB0 jako linie wejsciowa
\item \code{Do} - Poczatek petli programu
\item \code{SET} - Ustaw wartosc na porcie 1
\item \code{RESET} - Ustaw wartosc na porcie na 0
\item \code{TOGGLE} - zamien 1 na 0 i 0 na 1
\item \code{ROTATE PORTB.LEFT} - przesun wartosci w \code{PORTB} w lewo
\item \code{WAITMS 200} - opoznij dzialanie programu o 200ms
\item \code{LOOP} - Koniec glownej petli programu
\item \code{END} - Koniec programu 
\end{itemize}

\section{Wykorzystywane przyrzady}
\begin{itemize}
  \item Oscyloskop cyforwy
\end{itemize}

\begin{lstlisting}[language=VBScript , caption=Instrukcje Set i Reset]
  $regfile = "m8def.dat"
  $crystal  =  8000000  
  
  Config PINB.0  =  Output
  Do
    Set Portb.0
    Reset PORTB.0
  Loop
  End     
\end{lstlisting}

\begin{lstlisting}[language=VBScript , caption=Instrukcje Toggle]
  $regfile = "m8def.dat"
  $crystal  =  8000000
  Config PINB.0  =  Output
  Do                                    
    Tooggle PINB.0
  Loop
  End 
\end{lstlisting}

Dla programow napisanych w jezyku programowania BASCOM uzywajac symulatora mozemy odczytac ilosc taktow zegara jaka byla wymagana dla wykoanania danej instrukcji (dla polecenia \code{Toggle} sa to 4 cykle a \code{Reset} i \code{Set} zajmuja 2 takty zegara kazda) mierzac czasy wykonania obu implementacji otrzymujemy:

Dla polecen \code{Set} i \code{Reset}:

\begin{equation}
  \begin{split}
    T &= 252 ns + 500 ns = 752 ns,\\
    f &= \frac{1}{T} = \frac{1}{752ns} = 1,3297MHz
  \end{split}
\end{equation}

Dla polecenia \code{Toggle} natomiast:

\begin{equation}
  \begin{split}
    T &= 744ns + 744ns  = 1488ns,\\
    f &= \frac{1}{T} = \frac{1}{1488ns} = 0,62720MHz
  \end{split}
\end{equation}




\begin{lstlisting}[language=VBScript , caption=Implementacja w assemblerze]
  .nolist
  .include"m8def.inc"
  ldi r16, 0b00000001
  out ddrb, r16
  ldi r17, 0b00000000
      loop:
          out portb,r16
          out portb,r17
      rjmp loop
  .exit
\end{lstlisting}

Pomiary dla programu napisanego za pomoca assemblera mozemy niestety odczytac tylko za pomoca oscyloskopu cyfrowego

mozemy wtedy zaobserwowac ze czas trwania logicznej $1$ wynosi $128ns$ natomiast logicznego $0$ $360ns$

\begin{equation}
  T = 128ns + 360ns = 480ns
\end{equation}


\begin{equation}
  f = \frac{1}{T} = \frac{1}{488ns} = 2.05MHz
\end{equation}


\section{Podsumowanie}

Jak bardzo latwo zauwazyc analizujac otrzymane wyniki najwolniejsza z dostepnych dla nas opcji jest instrukcja \code{Toggle} wbudowana w jezyk BASCOM najszybsza metoda jest natomiast napisanie programu w czytstym Assemblerze. Musimy wiec balansowac poziom skompliowania programu i jego predkosc jego dzialania.

\end{document}
