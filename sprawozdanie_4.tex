\documentclass{article}
\usepackage{amsmath}
\usepackage{graphicx}
\usepackage{hyperref}
\usepackage[utf8]{inputenc}

\usepackage{listings}
\usepackage{color}

\NewDocumentCommand{\codeword}{v}{
\texttt{\textcolor{black}{#1}}
}

\definecolor{dkgreen}{rgb}{0,0.6,0}
\definecolor{gray}{rgb}{0.5,0.5,0.5}
\definecolor{mauve}{rgb}{0.58,0,0.82}

\lstset{frame=tb,
  aboveskip=3mm,
  belowskip=3mm,
  showstringspaces=false,
  columns=flexible,
  basicstyle={\small\ttfamily},
  numbers=none,
  numberstyle=\tiny\color{gray},
  keywordstyle=\color{blue},
  commentstyle=\color{dkgreen},
  stringstyle=\color{mauve},
  breaklines=true,
  breakatwhitespace=true,
  tabsize=3
}


\title{Sprawozdanie nr. 4}
\author{Mateusz Kojro}



\date{2020–01–13}
\begin{document}
\maketitle

\section{Opis cwiczenia}

Celem cwiczenia jest zaznajomienie z dostepnymi mozliwosciami odbierania danych wejsciowych od uzytkownika
Do mikrokontrolera mozemy podlaczyc guziki na sposoby

\begin{itemize}
  \item Bezposrednio do polaczen mikrokontrolera
  \item Matrycowo za pomoca odpowiedniego multipleksera - umozliwia podlaczenie wiekszej ilosci guzikow do tej samej liczby wejsc mikrokontrolera
\end{itemize}

\section{Intrukcje jezyka BASCOM wymagane do obslugi przyciskow}

\begin{itemize}
  \item \codeword{Debounce} - Uruchamia dany kod po nacisnieciu przycisku
  \item \codeword{Getatkbd()} - pobiera kod z klawiatury i odczytuje jego wartosci ASCII z predeklarowanej tablicy 
\end{itemize}

\section{Wykorzystane elementy elektroniczne}

\begin{itemize}
  \item Diody LED
  \item Przyciski
  \item Wyswietlacz LCD
\end{itemize}


\section{Przykladowe implementacje}


\begin{lstlisting}[language=VBScript , caption=Obsluga 2 przyciskow za pomoca Debounce]
  $regfile = "m8def.dat"
  $crystal = 8000000
  Config Lcd = 16 * 2

  Config PINB.1 = Input
  Config PINB.2 = Input
  Config PINB.3 = Output
  Config PINB.4 = Output

  S1 Alias PINB.2
  S2 Alias PINB.1

  Led1 Alias PINB.5
  Led2 Alias PINB.4

  Set PORTB.1
  Set PORTB.2

  Do
    Debounce S1, 0, Prog1, Sub
    Debounce S2, 0, Prog2, Sub
  Loop
:
  Prog1:
    Toggle Led1
  Return

  Prog1:
    Toggle Led1
  Return
\end{lstlisting}


\begin{lstlisting}[language=VBScript , caption=Obsluga klawiatury matrycowej]
  $regfile = "m8def.dat"
  $crystal = 8000000
  
  Config PINB.0 = Input
  Config PINB.1 = Input
  
  Config PINB.2 = Output
  Config PINB.3 = Output
  
  Config Lcd = 16*2
  Config Timer0 = Timer , Prescale = 1024 
  
  On Timer0=multiplex
  
  Dim t1 As Byte 
  Dim t2 As Byte 
  
  Dim button As Byte 
  Dim i As Byte 
  
  row1 Alias PINB.0
  row2 Alias PINB.1
  
  k1 Alias PINB.2
  k2 Alias PINB.3
  
  Enable Interrupts
  Enable Timer0
  La Timer0, 200
  
  Set PORTB.0
  Set PORTB.1
  Set k1
  Set k2
  
  Do
      Cls
      Lcd button
      Waitms 200
  Loop
  End
  
  multiplex:
      Load Timer0, 200
      For i = 1 To 2
         If i = 1 Then
             Reset k1
         Else 
             Set k1
             Reset k2
         End If
         If row1 = 0 Or row2 = 0 Then
             t1 = PINB And &B00000011
             Exit For
         Else
             T1=0
         End If 
      Next i
      IF t2 = t1 Then
          button = t1
          IF i=2 Then 
              button = button + 2
          End IF
      Else 
         t2 = t1
      End IF
  
      Set k1
      Ser k2 
  Return
\end{lstlisting}


\section{Podsumowanie}

Przyjmowanie infromacji od uzytkownika jest jednym z podstawwowych wymagan aby budowane uklady znalazly zastosowanie
w rzeczywistych sytuacjach dlatego tez bardzo wazna jest umiejetnosc laczenia ich zarowno bezposrednio
dla malej ilosci guzikow jak i posrednio przez klawiatury matrycowe przy wiekszych ilosciach potrzebnych guzikow

\end{document}

