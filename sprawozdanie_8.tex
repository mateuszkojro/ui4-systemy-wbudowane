\documentclass{article}
\usepackage{amsmath}
\usepackage{graphicx}
\usepackage{hyperref}
\usepackage[utf8]{inputenc}

\usepackage{listings}
\usepackage{color}

\NewDocumentCommand{\codeword}{v}{%
\texttt{\textcolor{black}{#1}}%
}

\definecolor{dkgreen}{rgb}{0,0.6,0}
\definecolor{gray}{rgb}{0.5,0.5,0.5}
\definecolor{mauve}{rgb}{0.58,0,0.82}

\lstset{frame=tb,
  aboveskip=3mm,
  belowskip=3mm,
  showstringspaces=false,
  columns=flexible,
  basicstyle={\small\ttfamily},
  numbers=none,
  numberstyle=\tiny\color{gray},
  keywordstyle=\color{blue},
  commentstyle=\color{dkgreen},
  stringstyle=\color{mauve},
  breaklines=true,
  breakatwhitespace=true,
  tabsize=3
}


\title{Sprawozdanie nr. 8}
\author{Mateusz Kojro}


\date{2020–01–13}
\begin{document}
\maketitle


\section{Opis cwiczenia}

Cwiczenie ma na celu poznanie metod pozwalajacych na nadawanie i odbieranie sygnalow za pomoca tzw. Podczerwieni
(Fale elektromagnetyczne o dlugosci fali wiekszej od swiatla widzialnego) za pomoca diod LED. 
Podczas cwiczena wykorzystywac bedziemy ukladu scalonego odbiornika podczerwieni wewnatrz takiego ukladu 
znajduja sie elementy odbierajace filtrujace i dekodujace sygnaly. Wiekszoc transmisji odbywajacych sie w podczerwieni wykorzystuje tzw kod RC5 lub kod NEC

\subsection{Kod RC5}

Kod RC5 sklada sie z 14 bitowych slow danych (1 bit trwa 1.778ms) zakodowanych w kodzie Manchester. Slowo w kodzie Manchester sklada sie z;

\begin{center}
  \begin{tabular}{ |c|c|c|c| } 
  \hline
  Sygnal startu & Bit kontrolny & Adres urzadzenia & Komenda   \\
  \hline
  2 Bity & 1 Bit & 5 Bitow & 6 Bitow\\ 
  \hline
  \end{tabular}
\end{center}

Calkowity czas przeslania takiej ramki wynosi $ 24.889ms $

\subsection{Kod NEC}

Innym stosowanym protokolem jest tzw protokol NEC w ktorym wykorzytsuje sie odleglosci impulsow Bitow
Logiczne 1 trwa $2.25ms$ natomiast logiczne 0 trwa $1.125ms$ i sklada sie z:

\begin{center}
  \begin{tabular}{ |c|c| } 
  \hline
  Adres & Polecenie \\
  \hline
  8 Bitow & 8 Bitow \\ 
  \hline
  \end{tabular}
\end{center}

Przy tym zarowno adres jak i polecenie przesylane jest 2 razy.

\section{Instrukcje jezyka BASCOM wykorzystywane podczas pracy z sygnalami w podczerwieni}

\begin{itemize}
  \item \codeword{Config Int0 = Low Level} - konfiguruje przerwania wywolane logicznym 0 na wejsciu \codeword{Int0}
  \item \codeword{Config Rc5 = Pind.2} - Konfiguracja lini odbiornika podczerwieni
  \item \codeword{Getrc5 (adress, command)} - pobierz adres i komende odebrana przez odbiornik podczerwieni
  \item \codeword{Debounce [button], 0, program, Sub} - unikajac zaklocen jezeli nacisnieto \codeword{button} wykonaj polecenie
  \item \codeword{Rc5send [toggle_bit], [address], [command] } - wyslanie wartosci zapisanych w zmiennych command i adress
\end{itemize}

\section{Wykorzystane komponenty}
\begin{itemize}
  \item odbiornik podczerwieni
  \item Pilot (np. od telewizora)
  \item Odbiornik VS1838
  \item Nadajnik IR w kodzie NEC
\end{itemize}

\section{Przykładowe kody źródłowe}

\begin{lstlisting}[language=VBScript , caption=Kod odbierajacy sygnal w kodzie RC5]
  $regfile = "m8def.dat"
  $crystal = 8000000
  Config Lcd = 16 * 2
  Config INT0 = Low Level
  Config Rc5 = PIND.2
  Dim addr As Byte, cmd As Byte
  Dim recived As Bit

  Set recived

  Do 
    If recived = 1 Then
      Cls
      Lcd "Addr: " ; addr
      Lowerline
      Lcd "Cmd: " ; cmd
      Reset recived
      Enable INT0
    End If
  Loop
  End
\end{lstlisting}


\begin{lstlisting}[language=VBScript , caption=Kod wysylajacy sygnal (przez podczerwien) w kodzie RC5 RC6 i Sony]
  S2 Alias PINC . 2
  S3 Alias PINC . 0
  Do
    Debounce S1, 0, Rc5, Sub
    Debounce S2, 0, Rc6, Sub
    Debounce S3, 0, Sony, Sub   
  Loop
  End

  Rc5:
    Command = 12
    Togbit = 0
    Adress = 0

    Do
      Rc5send Togbit, Address, Command
      Waitms 250
    Loop Untill S1 = 1
  Return

  Rc6:
    Command = 12
    Togbit = 0
    Adress = 0

    Do
      Rc6send Togbit, Address, Command
      Waitms 250
    Loop Untill S1 = 1
  Return

  Sony:
    Command = 12
    Togbit = 0
    Adress = 0

    Do
      Sonysend Togbit, Address, Command
      Waitms 250
    Loop Untill S1 = 1
  Return
\end{lstlisting}

\end{document}