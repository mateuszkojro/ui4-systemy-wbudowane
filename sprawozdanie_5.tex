\documentclass{article}
\usepackage{amsmath}
\usepackage{graphicx}
\usepackage{hyperref}
\usepackage[utf8]{inputenc}

\usepackage{listings}
\usepackage{color}

\NewDocumentCommand{\codeword}{v}{
\texttt{\textcolor{black}{#1}}
}

\definecolor{dkgreen}{rgb}{0,0.6,0}
\definecolor{gray}{rgb}{0.5,0.5,0.5}
\definecolor{mauve}{rgb}{0.58,0,0.82}

\lstset{frame=tb,
  aboveskip=3mm,
  belowskip=3mm,
  showstringspaces=false,
  columns=flexible,
  basicstyle={\small\ttfamily},
  numbers=none,
  numberstyle=\tiny\color{gray},
  keywordstyle=\color{blue},
  commentstyle=\color{dkgreen},
  stringstyle=\color{mauve},
  breaklines=true,
  breakatwhitespace=true,
  tabsize=3
}


\title{Sprawozdanie nr. 5}
\author{Mateusz Kojro}


\date{2020–01–13}
\begin{document}
\maketitle

\section{Opis cwiczenia}

Cwiczenie polega na projekcie konstrukcji prostego ukladu zegara opartego na 
przerwaniach zglaszanych podczas przepelnienia wbudowanego timera 

\section{Funkjce jezyka BASCOM potrzebne do implementacji zegara}

\begin{itemize}
  \item \codeword{Makebcd(value)} - przeksztalca \codeword{value} na wartosc w 2 cyforwym kdzoie BCD
  \item \codeword{Config Clock = Soft | User [, Gosub [Sectick] ] }: 
    \begin{itemize}
      \item \codeword{Soft} - Zliczanie odbywa sie automatycznie
      \item \codeword{User} - Zliczanie odbywa sie za pomoca zdefiniowanej przez uzytkownika funkcji
      \item \codeword{Gosub = Sectick} - Umozliwia automatyczne uruchamianie funkcji Settic w jednakowych odstepach czasu
    \end{itemize}
  \item \codeword{Config Date = Dmy | Mdy | Ymd , Separator = znak} - Wybiera uzywany fromat daty (D - dzien, M - miesiac, Y - rok)
  \codeword{znak} - znak oddzielajacy poszczegolne daty do wyboru - '/', '-', '.' 
\end{itemize}

\section{Wykorystywane elementy elektroniczne}
\begin{itemize}
  \item Guziki
  \item Wyswietlacz LCD
  \item Mikrokontroler z timerem asynchroniczym
\end{itemize}


\section{Kod zrodlowy implementujacy program zegara}


\begin{lstlisting}[language=VBScript , caption=Implementacja zegara oparteg ona przerwaniach]
  $regfile = "m8def.dat"
  $crystal = 8000000
  Config Lcd = 16 * 2

  Config PINB.1 = Input
  Config PINB.2 = Input

  Config Timer1 = Timer, Prescale = 256

  Declare Sub showtime
  On Timer1 count1s

  Dim S As Byte
  Dim M As Byte
  Dim H As Byte
  Dim new As Byte
  Dim bcd As Byte

  S1 Alias PINB.2
  S2 Alias PINB.1


  Enable Interupts 
  Enable Timer1
  Counter1 = 34286

  Set new 
  Set PORTB.1
  Set PORTB.2

  Do
    Call showtime
    If S1 = o Then
      Waitms 25
      If s1 = 0 Then
        Incr M
        S = 0
        if M = 60 Then
          M = 0
        End If
        Set new
        Call showtime
      End If
    End If
    If S2 = 0 Then
      Waitms 25
        If S2 = 0 Then
          Incr H
          if H = 24 Then
            H = 0
          End If
          Set new
          Call showtime
          Waitms 200
        End If
    End If
  Loop
  End
\end{lstlisting}



\end{document}
